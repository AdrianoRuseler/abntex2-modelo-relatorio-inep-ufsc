%% abtex2-modelo-relatorio-tecnico.tex, v-1.9.5 laurocesar
%% Copyright 2012-2015 by abnTeX2 group at http://www.abntex.net.br/ 
%%
%% This work may be distributed and/or modified under the
%% conditions of the LaTeX Project Public License, either version 1.3
%% of this license or (at your option) any later version.
%% The latest version of this license is in
%%   http://www.latex-project.org/lppl.txt
%% and version 1.3 or later is part of all distributions of LaTeX
%% version 2005/12/01 or later.
%%
%% This work has the LPPL maintenance status `maintained'.
%% 
%% The Current Maintainer of this work is the abnTeX2 team, led
%% by Lauro César Araujo. Further information are available on 
%% http://www.abntex.net.br/
%%
%% This work consists of the files abntex2-modelo-relatorio-tecnico.tex,
%% abntex2-modelo-include-comandos and abntex2-modelo-references.bib
%%


% ------------------------------------------------------------------------
% ------------------------------------------------------------------------
% abnTeX2: Modelo de Relatório Técnico/Acadêmico em conformidade com 
% ABNT NBR 10719:2011 Informação e documentação - Relatório técnico e/ou
% científico - Apresentação
% ------------------------------------------------------------------------ 
% ------------------------------------------------------------------------

\documentclass[
	% -- opções da classe memoir --
	12pt,				% tamanho da fonte
	openright,			% capítulos começam em pág ímpar (insere página vazia caso preciso)
	twoside,			% para impressão em verso e anverso. Oposto a oneside
	a4paper,			% tamanho do papel. 
	% -- opções da classe abntex2 --
	sumario=tradicional,
%	sumario=abnt-6027-2012, %  soh pode ser usado com memoir v3.6k ou superior
	%chapter=TITLE,		% títulos de capítulos convertidos em letras maiúsculas
	%section=TITLE,		% títulos de seções convertidos em letras maiúsculas
	%subsection=TITLE,	% títulos de subseções convertidos em letras maiúsculas
	%subsubsection=TITLE,% títulos de subsubseções convertidos em letras maiúsculas
	% -- opções do pacote babel --
	english,			% idioma adicional para hifenização
	brazil,				% o último idioma é o principal do documento
	]{abntex2-inep-ufsc-relatorio}


% ---
% PACOTES
% ---
% ---
% Pacotes de citações 
% ---

\usepackage[brazilian,hyperpageref]{backref}	 % Paginas com as citações na bibl
%\usepackage[alf]{abntex2cite}	% Citação alfabética por autor-data [alf]
\usepackage[num]{abntex2cite} % Citação numérica [num]
\citebrackets[]

% --- 
% CONFIGURAÇÕES DE PACOTES
% --- 

\renewcommand{\lstlistingname}{Código--fonte }% Listing -> Codigo fonte
\renewcommand{\lstlistlistingname}{Lista de códigos--fonte}% List of Listings -> Lista de códigos-fonte
\newlistof{lstlistoflistings}{lol}{\lstlistlistingname}

% ---
% Configurações do pacote backref
% Usado sem a opção hyperpageref de backref
\renewcommand{\backrefpagesname}{Citado na(s) página(s):~}
% Texto padrão antes do número das páginas
\renewcommand{\backref}{}
% Define os textos da citação
\renewcommand*{\backrefalt}[4]{
	\ifcase #1 %
		Nenhuma citação no texto.%
	\or
		Citado na página #2.%
	\else
		Citado #1 vezes nas páginas #2.%
	\fi}%
% ---

% ---
% Informações de dados para CAPA e FOLHA DE ROSTO
% ---
\titulo{Modelo Canônico de Relatórios técnicos com \abnTeX}
\autor{Nome do Autor 01 \\ Nome do Autor 02 \\ Nome do Autor 03 \\ Nome do Autor 04 \\ Nome do Autor 05}
\local{Florianópolis, Santa Catarina -- Brasil}
%\local{Florianópolis}
\data{\today}
\instituicao{%
	Universidade Federal de Santa Catarina -- UFSC
	\par
	Departamento de Engenharia Elétrica -- EEL
	\par
	Programa de Pós--Graduação em Engenharia Elétrica -- PGEEL}
\tipotrabalho{Relatório técnico}
% O preambulo deve conter o tipo do trabalho, o objetivo, 
% o nome da instituição e a área de concentração 
\preambulo{Modelo canônico de Relatório Técnico e/ou Científico em conformidade
com as normas ABNT apresentado à comunidade de usuários \LaTeX.}
% ---


\universidade{Universidade Federal de Santa Catarina}
\centro{Centro Tecnológico}
\departamento{Departamento de Engenharia Elétrica e Eletrônica}
\programa{Programa de Pós--graduação em Engenharia Elétrica}

\laboratorio{Instituto de Eletrônica de Potência} 



% ---
% Configurações de aparência do PDF final

%
%\definecolor{figcolor}{rgb}{1,0.4,0}  % orange
%\definecolor{tabcolor}{rgb}{1,0.4,0}  % orange
%\definecolor{eqncolor}{rgb}{1,0.4,0}  % orange
%\definecolor{linkcolor}{rgb}{1,0.4,0}  % orange
%\definecolor{citecolor}{rgb}{1,0.4,0}  % orange
%\definecolor{seccolor}{rgb}{0,0,1}  % blue
%\definecolor{abscolor}{rgb}{0,0,1}  % blue
%\definecolor{titlecolor}{rgb}{0,0,1}  % blue
%\definecolor{biocolor}{rgb}{0,0,1}  % blue

% alterando o aspecto da cor azul
\definecolor{blue}{RGB}{41,5,195}

% informações do PDF
\makeatletter
\hypersetup{
	%pagebackref=true,
	pdftitle={\@title}, 
	pdfauthor={Autores},
	pdfsubject={Relatório Técnico},
	pdfcreator={LaTeX with abnTeX2},
	pdfkeywords={abnt}{latex}{UFSC}{abntex2}{tese}{INEP}, 
	colorlinks=true,       		% false: boxed links; true: colored links
	linkcolor=linkcolor,          	% color of internal links
	citecolor=citecolor,        		% color of links to bibliography
	filecolor=black,      		% color of file links
	urlcolor=linkcolor,
	bookmarksdepth=4
}
\makeatother
% --- 



% ---
% Estilo de capítulos
%
%\chapterstyle{default}
% \chapterstyle{pedersen} 
%\chapterstyle{lyhne} 
%\chapterstyle{madsen} 
% \chapterstyle{veelo} 
%\chapterstyle{companion}

%\chapterstyle{thatcher}
%\chapterstyle{verville}

\chapterstyle{VZ14} % Ver classe para maiores detalhes


%
% Veja outros estilos em:
% http://www.tex.ac.uk/tex-archive/info/MemoirChapStyles/MemoirChapStyles.pdf
% ---

% http://tex.stackexchange.com/questions/228936/setting-entries-of-list-of-listings-in-latex-package-listings
\newlength\mylen

\begingroup
\makeatletter
\let\newcounter\@gobble\let\setcounter\@gobbletwo
\globaldefs\@ne \let\c@loldepth\@ne
\newlistof{listings}{lol}{\lstlistlistingname}
\newlistentry{lstlisting}{lol}{0}
\makeatother
\endgroup

\renewcommand\cftlstlistingpresnum{\lstlistingname~}
\settowidth\mylen{\cftlstlistingpresnum\cftlstlistingaftersnum}
\addtolength\cftlstlistingnumwidth{\mylen} %
\renewcommand\cftlstlistingaftersnum{\hfill\textendash\hfill}


% --- 
% Espaçamentos entre linhas e parágrafos 
% --- 

% O tamanho do parágrafo é dado por:
\setlength{\parindent}{1.3cm}

% Controle do espaçamento entre um parágrafo e outro:
\setlength{\parskip}{0.2cm}  % tente também \onelineskip

% ---
% compila o indice
% ---
\makeindex
% ---


%\includeonly{PreTexto/fichacatalografica}
%\includeonly{PreTexto/agradecimentos}
%\includeonly{PreTexto/resumos}
%\includeonly{PreTexto/siglas}
%\includeonly{PreTexto/simbolos}
%
%\includeonly{Capitulos/00/CH00}
%\includeonly{Capitulos/01/CH01}
%\includeonly{Capitulos/02/CH02}
%\includeonly{Capitulos/03/CH03}
%\includeonly{Capitulos/04/CH04}




% ----
% Início do documento
% ----
\begin{document}

% Seleciona o idioma do documento (conforme pacotes do babel)
%\selectlanguage{english}
\selectlanguage{brazil}

% Retira espaço extra obsoleto entre as frases.
\frenchspacing 

% ----------------------------------------------------------
% ELEMENTOS PRÉ-TEXTUAIS
% ----------------------------------------------------------
% \pretextual

% ---
% Capa
% ---
\imprimircapaUFSC  % Aplica o modelo com logo da UFSC
%\imprimircapa % Aplica o modelo default do pacote abntex2
% ---



% ---
% Folha de rosto
% (o * indica que haverá a ficha bibliográfica)
% ---
% \imprimirfolhaderosto
% ---


% ---
% Agradecimentos
% ---
%\input{PreTexto/agradecimentos.tex}
% ---

% ---
% RESUMO
% ---

% resumo na língua vernácula (obrigatório)
\setlength{\absparsep}{18pt} % ajusta o espaçamento dos parágrafos do resumo

\input{PreTexto/resumos}

% ---

% ---
% inserir lista de ilustrações
% ---
\pdfbookmark[0]{\listfigurename}{lof}
\listoffigures*
\cleardoublepage
% ---

% ---
% inserir lista de tabelas
% ---
\pdfbookmark[0]{\listtablename}{lot}
\listoftables*
\cleardoublepage
% ---


% ---
% inserir códigos fonte
% ---
\pdfbookmark[0]{\lstlistingname}{lol}
\lstlistoflistings*
\cleardoublepage
% ---

% ---
% inserir lista de abreviaturas e siglas
% ---
\input{PreTexto/siglas}
% ---

% ---
% inserir lista de símbolos
% ---
\input{PreTexto/simbolos}
% ---



% ---
% inserir o sumario
% ---
\pdfbookmark[0]{\contentsname}{toc}
\tableofcontents*
\cleardoublepage
% ---


% ----------------------------------------------------------
% ELEMENTOS TEXTUAIS
% ----------------------------------------------------------

\textual

%\textualINEPUFSC

% ----------------------------------------------------------
% Introdução (exemplo de capítulo sem numeração, mas presente no Sumário)
% ----------------------------------------------------------

% ----------------------------------------------------------
% Introdução (exemplo de capítulo sem numeração, mas presente no Sumário)
% ----------------------------------------------------------
\include{Capitulos/00/CH00}



% ----------------------------------------------------------
% PARTE - preparação da pesquisa
% ----------------------------------------------------------
\part{Preparação do relatório}

% ----------------------------------------------------------
% Capitulo com exemplos de comandos inseridos de arquivo externo 
% ----------------------------------------------------------

\include{Capitulos/01/CH01}
%\include{abntex2-modelo-include-comandos}



% ----------------------------------------------------------
% Parte de revisão e literatura
% ----------------------------------------------------------
\part{Resultados}

% ---
% Capitulo de revisão de literatura
% ---
\include{Capitulos/02/CH02}
% ---

% ---
% Finaliza a parte no bookmark do PDF
% para que se inicie o bookmark na raiz
% e adiciona espaço de parte no Sumário
% ---
\phantompart

% ---
% Conclusão
% ---

\include{Capitulos/Concl/Concl}

% ----------------------------------------------------------
% ELEMENTOS PÓS-TEXTUAIS
% ----------------------------------------------------------
\postextual

% ----------------------------------------------------------
% Referências bibliográficas
% ----------------------------------------------------------

\bibliographystyle{abnt-alf}
%\bibliographystyle{abnt-num}
\bibliography{abntex2-modelo-references}

% ----------------------------------------------------------
% Glossário
% ----------------------------------------------------------
%
% Consulte o manual da classe abntex2 para orientações sobre o glossário.
%
%\glossary

% ----------------------------------------------------------
% Apêndices
% ----------------------------------------------------------

% ---
% Inicia os apêndices
% ---
\begin{apendicesenv}

% Imprime uma página indicando o início dos apêndices
\partapendices

\include{Apendices/A/APA}
\include{Apendices/B/APB}
\include{Apendices/C/APC}

\end{apendicesenv}
% ---


% ----------------------------------------------------------
% Anexos
% ----------------------------------------------------------

% ---
% Inicia os anexos
% ---
\begin{anexosenv}

% Imprime uma página indicando o início dos anexos
\partanexos

\include{Anexos/A/ANA}
\include{Anexos/B/ANB}
\include{Anexos/C/ANC}

\end{anexosenv}




%---------------------------------------------------------------------
% INDICE REMISSIVO
%---------------------------------------------------------------------

\phantompart

\printindex



\end{document}
