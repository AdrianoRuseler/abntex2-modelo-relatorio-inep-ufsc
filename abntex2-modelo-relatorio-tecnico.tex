%% abtex2-modelo-relatorio-tecnico.tex, v-1.9.5 laurocesar
%% Copyright 2012-2015 by abnTeX2 group at http://www.abntex.net.br/ 
%%
%% This work may be distributed and/or modified under the
%% conditions of the LaTeX Project Public License, either version 1.3
%% of this license or (at your option) any later version.
%% The latest version of this license is in
%%   http://www.latex-project.org/lppl.txt
%% and version 1.3 or later is part of all distributions of LaTeX
%% version 2005/12/01 or later.
%%
%% This work has the LPPL maintenance status `maintained'.
%% 
%% The Current Maintainer of this work is the abnTeX2 team, led
%% by Lauro César Araujo. Further information are available on 
%% http://www.abntex.net.br/
%%
%% This work consists of the files abntex2-modelo-relatorio-tecnico.tex,
%% abntex2-modelo-include-comandos and abntex2-modelo-references.bib
%%


% ------------------------------------------------------------------------
% ------------------------------------------------------------------------
% abnTeX2: Modelo de Relatório Técnico/Acadêmico em conformidade com 
% ABNT NBR 10719:2011 Informação e documentação - Relatório técnico e/ou
% científico - Apresentação
% ------------------------------------------------------------------------ 
% ------------------------------------------------------------------------

\documentclass[
	% -- opções da classe memoir --
	12pt,				% tamanho da fonte
	openright,			% capítulos começam em pág ímpar (insere página vazia caso preciso)
	twoside,			% para impressão em verso e anverso. Oposto a oneside
	a4paper,			% tamanho do papel. 
	% -- opções da classe abntex2 --
	%chapter=TITLE,		% títulos de capítulos convertidos em letras maiúsculas
	%section=TITLE,		% títulos de seções convertidos em letras maiúsculas
	%subsection=TITLE,	% títulos de subseções convertidos em letras maiúsculas
	%subsubsection=TITLE,% títulos de subsubseções convertidos em letras maiúsculas
	% -- opções do pacote babel --
	english,			% idioma adicional para hifenização
	french,				% idioma adicional para hifenização
	spanish,			% idioma adicional para hifenização
	brazil,				% o último idioma é o principal do documento
	]{abntex2-inep-ufsc-relatorio}


% ---
% PACOTES
% ---

% ---
% Pacotes fundamentais 
% ---
\usepackage{lmodern}			% Usa a fonte Latin Modern
\usepackage[T1]{fontenc}		% Selecao de codigos de fonte.
\usepackage[utf8]{inputenc}		% Codificacao do documento (conversão automática dos acentos)
\usepackage{indentfirst}		% Indenta o primeiro parágrafo de cada seção.
\usepackage{color}				% Controle das cores
\usepackage{graphicx}			% Inclusão de gráficos
\usepackage{microtype} 			% para melhorias de justificação
% ---

% ---
% Pacotes adicionais, usados no anexo do modelo de folha de identificação
% ---

\usepackage{pdfpages}
\usepackage{fancyvrb}
\usepackage{rotating}
\usepackage{amsmath}		
\usepackage{amssymb}
\usepackage{mathrsfs}
\usepackage{color}
\usepackage{pdflscape}
\usepackage{epstopdf}
\usepackage{multirow}
\usepackage{listings}
\usepackage{url} % Para incluir links
\usepackage{nomencl} % Pacote necessário para a lista de siglas.
\usepackage{mathcomp}
\usepackage{booktabs} % Para Tabelas
\usepackage{subfig}  % permite ter subfiguras
\usepackage{float}

\usepackage{siunitx} % A comprehensive (SI) units package

\usepackage{datetime}

\usepackage{rotating} % gives you the possibility to rotate any object of an arbitrary angle. 
\usepackage{pdflscape} % Rotação de páginas no documento pdf.

	
% ---
% Pacotes adicionais, usados apenas no âmbito do Modelo Canônico do abnteX2
% ---
\usepackage{lipsum}				% para geração de dummy text
% ---

% ---
% Pacotes de citações 
% ---

\usepackage[brazilian,hyperpageref]{backref}	 % Paginas com as citações na bibl
\usepackage[alf]{abntex2cite}	% Citações padrão ABNT

% --- 
% CONFIGURAÇÕES DE PACOTES
% --- 

% ---
% Configurações do pacote backref
% Usado sem a opção hyperpageref de backref
\renewcommand{\backrefpagesname}{Citado na(s) página(s):~}
% Texto padrão antes do número das páginas
\renewcommand{\backref}{}
% Define os textos da citação
\renewcommand*{\backrefalt}[4]{
	\ifcase #1 %
		Nenhuma citação no texto.%
	\or
		Citado na página #2.%
	\else
		Citado #1 vezes nas páginas #2.%
	\fi}%
% ---

% ---
% Informações de dados para CAPA e FOLHA DE ROSTO
% ---
\titulo{Modelo Canônico de Relatórios do INEP com \abnTeX}
\autor{Nome do Autor 01\\Nome do Autor 02 \\Nome do Autor 03}
\local{Florianópolis, Santa Catarina -- Brasil}
%\local{Florianópolis}
\data{\today}
\instituicao{%
	Universidade Federal de Santa Catarina -- UFSC
	\par
	Departamento de Engenharia Elétrica -- EEL
	\par
	Programa de Pós--Graduação em Engenharia Elétrica -- PGEEL}
\tipotrabalho{Relatório técnico}
% O preambulo deve conter o tipo do trabalho, o objetivo, 
% o nome da instituição e a área de concentração 
\preambulo{Modelo canônico de Relatório Técnico e/ou Científico em conformidade
com as normas ABNT apresentado à comunidade de usuários \LaTeX.}
% ---

\disciplina{EEL7011 - Eletrônica Básica}
\professor{Adriano Ruseler}

%\newcommand{\universidade}{Universidade Federal de Santa Catarina (UFSC) \par Centro de Ciências Humanas, Letras e Artes (CCHLA)}
%\newcommand{\curso}{Geografia}
%\newcommand{\professores}{Nome de Aluno 1; Nome de Aluno 2; Nome de Aluno 3; Nome de Aluno 4; Nome de Aluno 5; Nome de Aluno 6}
%\newcommand{\disciplina}{Metodologia da Ciência}
%\newcommand{\tema}{A história da filosofia como desenvolvimento da ideia}
%\newcommand{\turma}{1º Periodo}
%% \newcommand{\data}{28/05/2014}
%\newcommand{\tempodeaula}{30 minutos}


% ---
% Configurações de aparência do PDF final

% alterando o aspecto da cor azul
\definecolor{blue}{RGB}{41,5,195}

% informações do PDF
\makeatletter
\hypersetup{
     	%pagebackref=true,
		pdftitle={\@title}, 
		pdfauthor={\@author},
    	pdfsubject={\imprimirpreambulo},
	    pdfcreator={LaTeX with abnTeX2},
		pdfkeywords={abnt}{latex}{abntex}{abntex2}{relatório técnico}, 
		colorlinks=true,       		% false: boxed links; true: colored links
    	linkcolor=blue,          	% color of internal links
    	citecolor=blue,        		% color of links to bibliography
    	filecolor=magenta,      		% color of file links
		urlcolor=blue,
		bookmarksdepth=4
}
\makeatother
% --- 


% --- 
% Espaçamentos entre linhas e parágrafos 
% --- 

% O tamanho do parágrafo é dado por:
\setlength{\parindent}{1.3cm}

% Controle do espaçamento entre um parágrafo e outro:
\setlength{\parskip}{0.2cm}  % tente também \onelineskip

% ---
% compila o indice
% ---
\makeindex
% ---

% ----
% Início do documento
% ----
\begin{document}

% Seleciona o idioma do documento (conforme pacotes do babel)
%\selectlanguage{english}
\selectlanguage{brazil}

% Retira espaço extra obsoleto entre as frases.
\frenchspacing 

% ----------------------------------------------------------
% ELEMENTOS PRÉ-TEXTUAIS
% ----------------------------------------------------------
% \pretextual

% ---
% Capa
% ---
\imprimircapa
% ---



% ---
% Folha de rosto
% (o * indica que haverá a ficha bibliográfica)
% ---
\imprimirfolhaderosto*
% ---

% ---
% Anverso da folha de rosto:
% ---

{
\ABNTEXchapterfont

\vspace*{\fill}

Conforme a ABNT NBR 10719:2011, seção 4.2.1.1.1, o anverso da folha de rosto
deve conter:

\begin{alineas}
  \item nome do órgão ou entidade responsável que solicitou ou gerou o
   relatório; 
  \item título do projeto, programa ou plano que o relatório está relacionado;
  \item título do relatório;
  \item subtítulo, se houver, deve ser precedido de dois pontos, evidenciando a
   sua subordinação ao título. O relatório em vários volumes deve ter um título
   geral. Além deste, cada volume pode ter um título específico; 
  \item número do volume, se houver mais de um, deve constar em cada folha de
   rosto a especificação do respectivo volume, em algarismo arábico; 
  \item código de identificação, se houver, recomenda-se que seja formado
   pela sigla da instituição, indicação da categoria do relatório, data,
   indicação do assunto e número sequencial do relatório na série; 
  \item classificação de segurança. Todos os órgãos, privados ou públicos, que
   desenvolvam pesquisa de interesse nacional de conteúdo sigiloso, devem
    informar a classificação adequada, conforme a legislação em vigor; 
  \item nome do autor ou autor-entidade. O título e a qualificação ou a função
   do autor podem ser incluídos, pois servem para indicar sua autoridade no
   assunto. Caso a instituição que solicitou o relatório seja a mesma que o
   gerou, suprime-se o nome da instituição no campo de autoria; 
  \item local (cidade) da instituição responsável e/ou solicitante; NOTA: No
   caso de cidades homônimas, recomenda-se o acréscimo da sigla da unidade da
   federação.
  \item ano de publicação, de acordo com o calendário universal (gregoriano),
  deve ser apresentado em algarismos arábicos.
\end{alineas}

\vspace*{\fill}
}

% ---
% Agradecimentos
% ---
\input{PreTexto/agradecimentos.tex}
% ---

% ---
% RESUMO
% ---

% resumo na língua vernácula (obrigatório)
\setlength{\absparsep}{18pt} % ajusta o espaçamento dos parágrafos do resumo

\input{PreTexto/resumos.tex}

% ---

% ---
% inserir lista de ilustrações
% ---
\pdfbookmark[0]{\listfigurename}{lof}
\listoffigures*
\cleardoublepage
% ---

% ---
% inserir lista de tabelas
% ---
\pdfbookmark[0]{\listtablename}{lot}
\listoftables*
\cleardoublepage
% ---


% ---
% inserir lista de abreviaturas e siglas
% ---
\input{PreTexto/siglas.tex}
% ---

% ---
% inserir lista de símbolos
% ---
\input{PreTexto/simbolos.tex}
% ---

% ---
% inserir o sumario
% ---
\pdfbookmark[0]{\contentsname}{toc}
\tableofcontents*
\cleardoublepage
% ---


% ----------------------------------------------------------
% ELEMENTOS TEXTUAIS
% ----------------------------------------------------------
\textual

% ----------------------------------------------------------
% Introdução (exemplo de capítulo sem numeração, mas presente no Sumário)
% ----------------------------------------------------------

% ----------------------------------------------------------
% Introdução (exemplo de capítulo sem numeração, mas presente no Sumário)
% ----------------------------------------------------------
\input{Capitulos/00/CH00.tex}



% ----------------------------------------------------------
% PARTE - preparação da pesquisa
% ----------------------------------------------------------
\part{Preparação do relatório}

% ----------------------------------------------------------
% Capitulo com exemplos de comandos inseridos de arquivo externo 
% ----------------------------------------------------------

\input{Capitulos/01/CH01.tex}
%\include{abntex2-modelo-include-comandos}



% ----------------------------------------------------------
% Parte de revisão e literatura
% ----------------------------------------------------------
\part{Resultados}

% ---
% Capitulo de revisão de literatura
% ---
\input{Capitulos/02/CH02.tex}
% ---

% ---
% Finaliza a parte no bookmark do PDF
% para que se inicie o bookmark na raiz
% e adiciona espaço de parte no Sumário
% ---
\phantompart

% ---
% Conclusão
% ---

\input{Capitulos/Concl/Concl.tex}

% ----------------------------------------------------------
% ELEMENTOS PÓS-TEXTUAIS
% ----------------------------------------------------------
\postextual

% ----------------------------------------------------------
% Referências bibliográficas
% ----------------------------------------------------------
\bibliography{abntex2-modelo-references}

% ----------------------------------------------------------
% Glossário
% ----------------------------------------------------------
%
% Consulte o manual da classe abntex2 para orientações sobre o glossário.
%
%\glossary

% ----------------------------------------------------------
% Apêndices
% ----------------------------------------------------------

% ---
% Inicia os apêndices
% ---
\begin{apendicesenv}

% Imprime uma página indicando o início dos apêndices
\partapendices

\input{Apendices/A/APA.tex}
\input{Apendices/B/APB.tex}
\input{Apendices/C/APC.tex}

\end{apendicesenv}
% ---


% ----------------------------------------------------------
% Anexos
% ----------------------------------------------------------

% ---
% Inicia os anexos
% ---
\begin{anexosenv}

% Imprime uma página indicando o início dos anexos
\partanexos

\input{Anexos/A/ANA.tex}
\input{Anexos/B/ANB.tex}
\input{Anexos/C/ANC.tex}

\end{anexosenv}




%---------------------------------------------------------------------
% INDICE REMISSIVO
%---------------------------------------------------------------------

\phantompart

\printindex


%---------------------------------------------------------------------
% Formulário de Identificação (opcional)
%---------------------------------------------------------------------
\chapter*[Formulário de Identificação]{Formulário de Identificação}
\addcontentsline{toc}{chapter}{Exemplo de Formulário de Identificação}
\label{formulado-identificacao}

Exemplo de Formulário de Identificação, compatível com o Anexo A (informativo)
da ABNT NBR 10719:2011. Este formulário não é um anexo. Conforme definido na
norma, ele é o último elemento pós-textual e opcional do relatório.

\bigskip

\begin{tabular}{|p{9cm}|p{5cm}|}
\hline
\multicolumn{2}{|c|}{\textbf{\large Dados do Relatório Técnico e/ou científico}}\\
\hline
\multirow{4}{10cm}[24pt]{Título e subtítulo}& Classificação de segurança\\
                   & \\
                   \cline{2-2}
                   & No.\\
                   & \\
				
\hline
Tipo de relatório & Data\\
\hline
Título do projeto/programa/plano & No.\\
\hline
\multicolumn{2}{|l|}{Autor(es)} \\
\hline
\multicolumn{2}{|l|}{Instituição executora e endereço completo} \\
\hline
\multicolumn{2}{|l|}{Instituição patrocinadora e endereço completo} \\
\hline
\multicolumn{2}{|l|}{Resumo}\\[3cm]
\hline
\multicolumn{2}{|l|}{Palavras-chave/descritores}\\
\hline
\multicolumn{2}{|l|}{
Edição \hfill No. de páginas \hfill No. do volume \hfill Nº de classificação \phantom{XXXX}} \\
\hline
\multicolumn{2}{|l|}{
ISSN \hfill \hfill Tiragem \hfill Preço \phantom{XXXXXXXX}} \\
\hline
\multicolumn{2}{|l|}{Distribuidor} \\
\hline
\multicolumn{2}{|l|}{Observações/notas}\\[3cm]
\hline
\end{tabular}

\end{document}
